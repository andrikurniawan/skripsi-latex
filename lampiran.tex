%-----------------------------------------------------------------------------%
\addChapter{Lampiran 1 : Kode Sumber Model ABS}
\chapter*{Lampiran 1 : Kode Sumber Model ABS}
%-----------------------------------------------------------------------------%
\section*{\code{m\_abs.erl}} \label{cha:lampir-abs}
Berkas ini ditaruh pada \co{src/models/}. Berkas ini merupakan \textit{adaptor} yang akan menghubungkan antara ontologi dan \textit{web services}
\begin{lstlisting}[style=L,caption={Berkas adaptor m\_abs.erl},label={lst:mabs}]
%% @author Andri Kurniawan <andrikurniawan.id@gmail.com>
%% @copyright 2017 Andri Kurniawan
%% Date: 2017-05-11
%%
%% @doc Template access for abs model

%% Copyright 2017 Andri Kurniawan
%%
%% Licensed under the Apache License, Version 2.0 (the "License");
%% you may not use this file except in compliance with the License.
%% You may obtain a copy of the License at
%%
%%     http://www.apache.org/licenses/LICENSE-2.0
%%
%% Unless required by applicable law or agreed to in writing, software
%% distributed under the License is distributed on an "AS IS" BASIS,
%% WITHOUT WARRANTIES OR CONDITIONS OF ANY KIND, either express or implied.
%% See the License for the specific language governing permissions and
%% limitations under the License.
-module(m_abs).
-behaviour(gen_model).

-export([
	m_find_value/3,
	m_to_list/2,
	m_value/2,
	call_api_controller/2
]).

-include_lib("zotonic.hrl").

-define(RULES, "/home/andri/skripsi/zotonic/rules.txt").

% this method to handle call api from template
-spec m_find_value(Key, Source, Context) -> #m{} | undefined | any() when
    Key:: integer() | atom() | string(),
    Source:: #m{},
    Context:: #context{}.

m_find_value(Type, #m{value=undefined} = M, _Context) ->
    M#m{value=[Type]};

m_find_value({query, Query}, #m{value=Q} = _, _Context) when is_list(Q) ->
	[Key] = Q,
	[Url, Param] = lookup_rules(Key),
	case validate_params(Param, Query) of
		false ->
			[{error, "Num of Params not same"}];
		true ->
			{DecodeJson} = fetch_data(binary_to_list(Url), jiffy:encode({Query})),
			lager:info("ABS result : ~p", [DecodeJson]),
			proplists:get_value(<<"data">>, DecodeJson)
	end;

% Other values won't be processed
m_find_value(_, _, _Context) ->
    undefined. 

m_to_list(_, _Context) ->
	[].

m_value(_, _Context) ->
	undefined.

% this method to handle call api from another module
call_api_controller(Key, Data) ->
	[Url, Param] = lookup_rules(Key),
	case validate_params(Param, Data) of
		false ->
			[{error, "Num of Params not same"}];
		true ->
			lager:info("key ~p", [Data]),
			lager:info("key ~s", [jiffy:encode({Data})]),
			{DecodeJson} = fetch_data(binary_to_list(Url), jiffy:encode({Data})),
			lager:info("[ABS] result ~p", [DecodeJson]),
			case proplists:get_value(<<"status">>, DecodeJson) of
				200 ->
					DataResult = proplists:get_value(<<"data">>, DecodeJson),
					lager:info("[ABS] status 200 ~p", [DataResult]);
				201 ->
					Message = proplists:get_value(<<"message">>, DecodeJson),
					lager:info("[ABS] status 201 ~p", [binary_to_list(Message)]);
				400 ->
					Message = proplists:get_value(<<"message">>, DecodeJson),
					lager:error("[ABS] status 400 ~p", [Message]);
				_Other -> 
					lager:error("[ABS] status undefined ~p", [_Other])
			end  
	end.

-spec fetch_data(Url, Query) -> list() when
	Url:: list(),
    Query:: list().
fetch_data(_,[]) ->
    [{error, "Params missing"}];
fetch_data("",_) ->
	[{error, "Url missing"}];
fetch_data(Url, Query) ->
	    case post_page_body(Url, Query) of
	        {error, Error} ->
	            [{error, Error}];
	        Json ->     
	            jiffy:decode(Json)
	    end.

post_page_body(Url, Body) ->
	case httpc:request(post, {Url, [], "application/json", Body}, [], []) of
		{ok,{_, _, Response}} ->
			Response;
		Error ->
			{error, Error}
	end.

lookup_rules(Key) ->
	File = ?RULES,
	case read_file(File) of
		{error, Error} ->
			[{error, Error}];
		[] ->
			[{error, "File empty"}];
		Json ->
			{DecodeJson} = jiffy:decode(Json),
			proplists:get_value(atom_to_binary(Key, latin1), DecodeJson)
	end.

read_file(File) ->
	case file:read_file(File) of
		{ok, Data} ->
			Data;
		eof ->
			[];
		Error ->
			{error, Error}
	end.

validate_params(Param, Query) ->
	case length(Query) == Param of
		false ->
			false;
		true ->
			true
end.
\end{lstlisting}

%-----------------------------------------------------------------------------%
\addChapter{Lampiran 2 : Kode Sumber rules}
\chapter*{Lampiran 2 : Kode Sumber rules}
%-----------------------------------------------------------------------------%
\section*{rules.txt}
Berkas ini ditaruh pada direktori \textit{root} dari zotonic. Berkas ini berfungsi sebagai tabel \textit{mapping}.
\begin{lstlisting}[caption={Berkas rules.txt},label={lst:excomp}]
{
	"createProgram": ["http://54.169.128.6:8080/abs/program/create", 2],
	"createDonation": ["http://54.169.128.6:8080/abs/donation/create", 4],
	"updateProgram": ["http://54.169.128.6:8080/abs/program/update", 2],
	"updateDonation": ["http://54.169.128.6:8080/abs/donation/update", 4],
	"deleteProgram": ["http://54.169.128.6:8080/abs/program/delete", 1],
	"deleteDonation": ["http://54.169.128.6:8080/abs/donation/delete", 1],
	"totalDonation" : ["http://54.169.128.6:8080/abs/program/total-donation", 1],
	"checkExistProgram" : ["http://54.169.128.6:8080/abs/program/checkExist", 1],
	"checkExistDonation" : ["http://54.169.128.6:8080/abs/donation/checkExist", 1]
}
\end{lstlisting}

%-----------------------------------------------------------------------------%
\addChapter{Lampiran 3 : Struktur Zotonic}
\chapter*{Lampiran 3 : Struktur Zotonic}
%-----------------------------------------------------------------------------%
Berikut adalah struktur dari Zotonic setelah implementasi penambahan berkas yang digunakan sebagai \textit{adaptor}.

\begin{tabbing}
	\qquad Zotonic/ \\
	\qquad \qquad .rebar/ \\
	\qquad \qquad bin/ \\
	\qquad \qquad deps/ \\
	\qquad \qquad doc/ \\
	\qquad \qquad docker/ \\
	\qquad \qquad ebin/ \\
	\qquad \qquad include/ \\
	\qquad \qquad modules/ \\
	\qquad \qquad priv/ \\
	\qquad \qquad src/ \\
	\qquad \qquad user/ \\
	\qquad \qquad .dockerignore \\
	\qquad \qquad .editorconfig \\
	\qquad \qquad .travis.yml \\
	\qquad \qquad AUTHORS \\
	\qquad \qquad CONTRIBUTING.md \\
	\qquad \qquad CONTRIBUTORS \\
	\qquad \qquad Dockerfile \\
	\qquad \qquad Dockerfile.dev \\
	\qquad \qquad Dockerfile.heavy \\
	\qquad \qquad GNUmakefile \\
	\qquad \qquad LICENSE\\
	\qquad \qquad Makefile \\
	\qquad \qquad Readme.md \\
	\qquad \qquad TRANSLATORS \\
	\qquad \qquad USE\_REBAR\_LOCKED \\
	\qquad \qquad build.cmd \\
	\qquad \qquad charity\_org\_rdf.owl \\
	\qquad \qquad classAndObjectPropertyMapper.sh \\
	\qquad \qquad docker-compose.yml \\
	\qquad \qquad prepare-release.sh \\
	\qquad \qquad rebar \\
	\qquad \qquad rebar.config\\
	\qquad \qquad rebar.config.lock\\
	\qquad \qquad rebar.config.lock.script\\
	\qquad \qquad rebar.config.script\\
	\qquad \qquad recentsite.txt\\
	\qquad \qquad start.cmd\\
	\qquad \qquad start.sh\\
	\qquad \qquad zotonic.pid\\
	\qquad \qquad zotonic\_install \\
	\qquad \qquad rules.txt
\end{tabbing}

%-----------------------------------------------------------------------------%
\addChapter{Lampiran 4 : Kode Sumber \textit{Web Service}}
\chapter*{Lampiran 4 : Kode Sumber \textit{Web Service}}
%-----------------------------------------------------------------------------%
\textit{Web Services} ini digunakan untuk melakukan ujicoba terhadap \textit{adaptor} yang telah diimplementasikan. \textit{Web Services} ini hanya bersifat sementara karena belum adanya \textit{web service} dari ABS \textit{microservices} untuk ontologi yang digunakan.
\section*{DonationController.java}
\begin{lstlisting}[caption={Berkas DonationController.java},label={lst:DonationController},language=Java]
package com.skripsi.Controller;

import com.skripsi.Domain.DonationEntity;
import com.skripsi.Domain.Wrapper;
import com.skripsi.Service.DonationService;
import org.springframework.beans.factory.annotation.Autowired;
import org.springframework.web.bind.annotation.*;

/**
* Created by Andri Kurniawan <andrikurniawan.id@gmail.com>
* on 5/13/2017.
*/

@RestController
@RequestMapping("/donation")
public class DonationController {

	@Autowired
	private DonationService donationService;
	
	@RequestMapping(path = "/create", method = RequestMethod.POST)
	public Wrapper create(@RequestBody DonationEntity donationEntity){
		DonationEntity donation = new DonationEntity();
		donation.setId(donationEntity.getId());
		donation.setName(donationEntity.getName());
		donation.setAmount(donationEntity.getAmount());
		donation.setpId(donationEntity.getpId());
		DonationEntity hasil = donationService.save(donation);
		
		if(hasil == null){
			return new Wrapper(400, "Gagal menyimpan donasi baru", null);
		}
		return new Wrapper(201, "Donasi berhasil disimpan", donation);
	}
	
	@RequestMapping(path = "/update", method = RequestMethod.POST)
	public Wrapper update(@RequestBody DonationEntity donationEntity) {
		DonationEntity donation = donationService.findById(donationEntity.getId());
		donation.setName(donationEntity.getName());
		donation.setAmount(donationEntity.getAmount());
		donation.setpId(donationEntity.getpId());
		DonationEntity hasil = donationService.save(donationEntity);
		
		if(hasil == null){
			return new Wrapper(400, "Gagal memperbaharui donasi", null);
		}
		return new Wrapper(201, "Donasi berhasil diperbaharui", donation);
	}
	
	@RequestMapping(path = "/delete", method = RequestMethod.POST)
	public Wrapper delete(@RequestBody DonationEntity donationEntity){
		int result = donationService.delete(donationEntity.getId());
		if (result == 0){
			return new Wrapper(400, "Gagal menghapus program", null);
		}
		return new Wrapper(201, "Berhasil menghapus program", result);
	}
}


\end{lstlisting}

\newpage
\section*{ProgramController.java}
\begin{lstlisting}[caption={Berkas ProgramController.java},label={lst:ProgramController},language=Java]
package com.skripsi.Controller;

import com.skripsi.Domain.ProgramEntity;
import com.skripsi.Domain.Wrapper;
import com.skripsi.Service.*;
import org.springframework.beans.factory.annotation.Autowired;
import org.springframework.web.bind.annotation.*;

import java.util.List;

/**
* Created by Andri Kurniawan <andrikurniawan.id@gmail.com>
* on 5/13/2017.
*/

@RestController
@RequestMapping("/program")
public class ProgramController {

	@Autowired
	private ProgramService programService;
	
	@Autowired
	private DonationService donationService;

	@RequestMapping(path = "/create", method = RequestMethod.POST)
	public Wrapper create(@RequestBody ProgramEntity programEntity){
		ProgramEntity program = new ProgramEntity();
		program.setId(programEntity.getId());
		program.setName(programEntity.getName());
		ProgramEntity hasil = programService.save(program);
		
		if(hasil == null){
			return new Wrapper(400, "Gagal menyimpan program baru", null);
		}
		return new Wrapper(201, "Program berhasil disimpan", program);
	}

	@RequestMapping(path = "/update", method = RequestMethod.POST)
	public Wrapper update(@RequestBody ProgramEntity programEntity){
		ProgramEntity program = programService.findById(programEntity.getId());
		program.setName(programEntity.getName());
		ProgramEntity hasil = programService.save(program);
		
		if(hasil == null){
			return new Wrapper(400, "Gagal memperbaharui program", null);
		}
		return new Wrapper(201, "Program berhasil diperbaharui", program);
	}

	@RequestMapping(path = "/delete", method = RequestMethod.POST)
	public Wrapper delete(@RequestBody ProgramEntity programEntity){
		int result = programService.delete(programEntity.getId());
		if (result == 0){
			return new Wrapper(400, "Gagal menghapus program", null);
		}
		return new Wrapper(201, "Berhasil menghapus program", result);
	}

	@RequestMapping(path = "/total-donation", method = RequestMethod.POST)
	public Wrapper totalDonation(@RequestBody ProgramEntity programEntity){
		int total = donationService.getTotalDonationOfProgram(programEntity.getId());
		if(total < 1){
			return new Wrapper(400, "Tidak ada donasi untuk program ini", 0);
		}
		return new Wrapper(200, "Berhasil", total);
	}
	
}

\end{lstlisting}

\newpage
\section*{DonationEntity.java}
\begin{lstlisting}[caption={Berkas DonationEntity.java},label={lst:DonationEntity},language=Java]
package com.skripsi.Domain;

import javax.persistence.*;

/**
* Created by Andri Kurniawan <andrikurniawan.id@gmail.com>
* on 5/13/2017.
*/
@Entity
@Table(name = "donation", schema = "abs_program")
public class DonationEntity {
	private int id;
	private String name;
	private int amount;
	private int pId;
	
	@Id
	@Column(name = "id")
	public int getId() {
		return id;
	}
	
	public void setId(int id) {
		this.id = id;
	}
	
	@Basic
	@Column(name = "name")
	public String getName() {
		return name;
	}
	
	public void setName(String name) {
		this.name = name;
	}
	
	@Basic
	@Column(name = "amount")
	public int getAmount() {
		return amount;
	}
	
	public void setAmount(int amount) {
		this.amount = amount;
	}
	
	@Basic
	@Column(name = "p_id")
	public int getpId() {
		return pId;
	}
	
	public void setpId(int pId) {
		this.pId = pId;
	}
	
	@Override
	public boolean equals(Object o) {
		if (this == o) return true;
		if (o == null || getClass() != o.getClass()) return false;
		
		DonationEntity that = (DonationEntity) o;
		
		if (id != that.id) return false;
		if (amount != that.amount) return false;
		if (pId != that.pId) return false;
		if (name != null ? !name.equals(that.name) : that.name != null) return false;
		
		return true;
	}
	
	@Override
	public int hashCode() {
		int result = id;
		result = 31 * result + (name != null ? name.hashCode() : 0);
		result = 31 * result + amount;
		result = 31 * result + pId;
		return result;
	}
}

\end{lstlisting}

\newpage
\section*{ProgramEntity.java}
\begin{lstlisting}[caption={Berkas ProgramEntity.java},label={lst:ProgramEntity},language=Java]
package com.skripsi.Domain;

import javax.persistence.*;

/**
* Created by Andri Kurniawan <andrikurniawan.id@gmail.com>
* on 5/13/2017.
*/
@Entity
@Table(name = "program", schema = "abs_program", catalog = "")
	public class ProgramEntity {
	private int id;
	private String name;
	
	@Id
	@Column(name = "id")
	public int getId() {
		return id;
	}
	
	public void setId(int id) {
		this.id = id;
	}
	
	@Basic
	@Column(name = "name")
	public String getName() {
		return name;
	}
	
	public void setName(String name) {
		this.name = name;
	}
	
	@Override
	public boolean equals(Object o) {
		if (this == o) return true;
		if (o == null || getClass() != o.getClass()) return false;
		
		ProgramEntity that = (ProgramEntity) o;
		
		if (id != that.id) return false;
		if (name != null ? !name.equals(that.name) : that.name != null) return false;
		
		return true;
	}
	
	@Override
	public int hashCode() {
		int result = id;
		result = 31 * result + (name != null ? name.hashCode() : 0);
		return result;
	}
}

\end{lstlisting}

\newpage
\section*{Wrapper.java}
\begin{lstlisting}[caption={Berkas Wrapper.java},label={lst:Wrapper},language=Java]
package com.skripsi.Domain;

/**
* Created by Andri Kurniawan <andrikurniawan.id@gmail.com>
* on 5/13/2017.
*/
public class Wrapper <E> {
	private int status;
	private String message;
	private E data;
	
	/**
	* Returns the status of this wrapper.
	* @return the {@code status} property
	*/
	public int getStatus() {
		return status;
	}
	
	/**
	* Sets the status of this wrapper with the specified {@code status}.
	* @param status the {@code status} for this wrapper
	*/
	public void setStatus(int status) {
		this.status = status;
	}
	
	/**
	* Returns the message of this wrapper.
	* @return the {@code message} property
	*/
	public String getMessage() {
		return message;
	}
	
	/**
	* Sets the message of this wrapper with the specified {@code message}.
	* @param message the {@code message} for this wrapper
	*/
	public void setMessage(String message) { this.message = message; }
	
	/**
	* Returns the aata of this wrapper.
	* @return the {@code data} property
	*/
	public E getData() { return data; }
	
	/**
	* Sets the data of this wrapper with the specified {@code data}.
	* @param data the {@code data} for this wrapper
	*/
	public void setData(E data) { this.data = data; }
	
	/**
	* Constructs a new Wrapper with the specified status, message, and data.
	* @param status the status for this wrapper
	* @param message the message for this wrapper
	* @param data the data for this wrapper
	*/
	public Wrapper(int status, String message, E data) {
	
		this.status = status;
		this.message = message;
		this.data = data;
	}
}

\end{lstlisting}

\newpage
\section*{DonationRepository.java}
\begin{lstlisting}[caption={Berkas DonationRepository.java},label={lst:DonationRepository},language=Java]
package com.skripsi.Repository;

import com.skripsi.Domain.DonationEntity;
import org.springframework.data.jpa.repository.Query;
import org.springframework.data.repository.CrudRepository;
import org.springframework.data.repository.query.Param;
import org.springframework.stereotype.Repository;

import javax.transaction.Transactional;
import java.util.List;

/**
* Created by Andri Kurniawan <andrikurniawan.id@gmail.com>
* on 5/13/2017.
*/
@Repository
public interface DonationRepository extends CrudRepository<DonationEntity, Integer> {
	DonationEntity findById(int id);
	List<DonationEntity> findBypId(int p_id);
	
	@Transactional
	Integer deleteById(int id);
	
	@Query(value = "Select sum(amount) from donation where p_id = :p_id",
	nativeQuery = true)
	Integer getTotalDonationOfProgram(@Param("p_id") int p_id);
}

\end{lstlisting}

\newpage
\section*{ProgramRepository.java}
\begin{lstlisting}[caption={Berkas ProgramRepository.java},label={lst:ProgramRepository},language=Java]
package com.skripsi.Repository;

import com.skripsi.Domain.ProgramEntity;
import org.springframework.data.repository.CrudRepository;
import org.springframework.stereotype.Repository;

import javax.transaction.Transactional;

/**
* Created by Andri Kurniawan <andrikurniawan.id@gmail.com>
* on 5/13/2017.
*/
@Repository
public interface ProgramRepository extends CrudRepository<ProgramEntity, Integer> {
	ProgramEntity findById(int id);
	@Transactional
	Integer deleteById(int id);
}

\end{lstlisting}

\newpage
\section*{DonationService.java}
\begin{lstlisting}[caption={Berkas DonationService.java},label={lst:DonationService},language=Java]
package com.skripsi.Service;

import com.skripsi.Domain.DonationEntity;
import com.skripsi.Repository.DonationRepository;
import org.springframework.beans.factory.annotation.Autowired;
import org.springframework.stereotype.Service;

import java.util.List;

/**
* Created by Andri Kurniawan <andrikurniawan.id@gmail.com>
* on 5/13/2017.
*/

@Service
public class DonationService {
	@Autowired
	private DonationRepository donationRepository;
	
	public DonationEntity findById(int id){
		return donationRepository.findById(id);
	}
	
	public int getTotalDonationOfProgram(int p_id){
		if (donationRepository.getTotalDonationOfProgram(p_id) == null)
			return 0;
		else
			return donationRepository.getTotalDonationOfProgram(p_id);
	}
	
	public List<DonationEntity> findByIdProgram(int p_id){
		return donationRepository.findBypId(p_id);
	}
	
	public DonationEntity save(DonationEntity donation){
		return donationRepository.save(donation);
	}
	
	public int delete(int id){
		return donationRepository.deleteById(id);
	}
}

\end{lstlisting}

\newpage
\section*{ProgramService.java}
\begin{lstlisting}[caption={Berkas ProgramService.java},label={lst:ProgramService},language=Java]
package com.skripsi.Service;

import com.skripsi.Domain.ProgramEntity;
import com.skripsi.Repository.ProgramRepository;
import org.springframework.beans.factory.annotation.Autowired;
import org.springframework.stereotype.Service;

import java.util.List;

/**
* Created by Andri Kurniawan <andrikurniawan.id@gmail.com>
* on 5/13/2017.
*/

@Service
public class ProgramService {
	@Autowired
	private ProgramRepository programRepository;
	
	public ProgramEntity findById(int id){
		return programRepository.findById(id);
	}
	
	public ProgramEntity save(ProgramEntity program){
		return programRepository.save(program);
	}
	
	public int delete(int id){
		return programRepository.deleteById(id);
	}
}

\end{lstlisting}