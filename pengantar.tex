%-----------------------------------------------------------------------------%
\chapter*{\kataPengantar}
%-----------------------------------------------------------------------------%
Segala puji dan syukur penulis ucapkan atas kehadirat Allah SWT, Tuhan Yang Maha Esa, karena atas rahmat dan karunia-Nya penulis dapat menyelesaikan skripsi yang berjudul "\judul". Pada kesempatan ini penulis ingin mengucapkan terima kasih kepada seluruh pihak yang telah membantu dan mendukung penulis selama proses pengerjaan skripsi ini, dimana berkat dukungan dan doa mereka skripsi ini dapat diselesaikan.

Penulisan skripsi ini ditujukan untuk memenuhi salah satu syarat untuk menyelesaikan pendidikan pada Program \gelar, Universitas Indonesia. Penulis sadar bahwa dalam perjalanan perkuliahan hingga penulisan skripsi ini, penulis tidak sendirian. Penulis ingin berterima kasih kepada pihak-pihak berikut :

\begin{enumerate}
\item Bapak Ade Azurat selaku dosen pembimbing tugas akhir yang telah meluangkan waktunya sepanjang semester ini untuk dapat memberikan arahan, kritik dan saran kepada penulis agar dapat menyelesaikan proses pengerjaan skripsi ini.
\item Bapak Drs. Lim Yohanes Stefanus M.Math., Ph.D selaku dosen pembimbing akademis penulis, yang selalu membantu penulis selama masa perkuliahan.
\item Drs. H. Zulhaspan, MM dan Hj. Masreni Nasution selaku orangtua dari penulis serta Anita Putri dan Akbar Syarif selaku saudara dari penulis yang selalu mendoakan, mendukung serta menjadi motivasi penulis dalam mengerjakan skripsi ini.
\item Kak Afifun yang telah memberikan banyak masukan terkait hal teknis kepada penulis.
\item Lab RSE
\item Nabila Akiti Hara selaku pacar dari penulis yang selalu memberikan motivasi dan mendukung penulis selama pengerjaan skripsi serta menemani penulis melalui \textit{facetime}.
\item Teman-teman PI BPH IMMM UI (Fadhil, Titto, Fandika, Mawan, Dodo, Okky, Devi, Rara, Ami, Rizky, Ime, Popo, Ilham) yang selalu menghibur penulis kala jenuh dalam mengerjakan skripsi dan seluruh keluarga IMMMSU UI yang telah menjadi keluarga bagi penulis selama masa perkuliahan.
\item Arief Radityo, Arsi Alhafis, dan M. Gibran yang selalu menjadi teman untuk bermain maupun belajar bagi penulis serta membantu penulis selama perkuliahan.
\item Sahabat-sahabat PPN (Abi, Budi, Cia, Dana, Erwin, Fakhry, Irene, Fadly, Fani, Mawan, Mutia, Sufi, Ulup) yang selalu menjadi penghibur bagi penulis setiap saat.
\item Teman-teman CornedIn (Arsi, Zaki, Dimas, Ilham) yang merupakan teman-teman perjuangan untuk proyekan yang mengajarkan banyak hal terkait teknikal kepada penulis.
\item Kelompok PPL B1 (Akbar, Dimas, Emon, Fajrin, Fathin), Kelompok PPL B2 (Gilang, Falah, Fatah, Nanda, Hamdan) dan Kelompok PPL B3 (Brigita, Gentur, Kowan, Riscel, Muthy) serta Kak Naya yang telah menemani penulis selama satu semester khususnya hari Rabu dan memberikan penulis pandangan baru mengenai \textit{scrum master}.
\end{enumerate}

Akhir kata, penulis berharap semoga Allah SWT dapat membalas kebaikan yang diberikan oleh orang-orang terdekat penulis dan penulis berharap karya yang penulis buat dapat membantu dan bermanfaat bagi pengembangan ilmu pengetahuan selanjutnya.

\vspace*{0.1cm}
\begin{flushright}
Depok, 5 Juni 2017\\[0.1cm]
\vspace*{1cm}
\penulis

\end{flushright}