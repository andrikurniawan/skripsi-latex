%
% Halaman Abstrak
%
% @author  Andreas Febrian
% @version 1.00
%

\chapter*{Abstrak}

\vspace*{0.2cm}

\noindent \begin{tabular}{l l p{10cm}}
	Nama&: & \penulis \\
	Program Studi&: & \program \\
	Judul&: & \judul \\
\end{tabular} \\ 

\vspace*{0.5cm}

\noindent 
\\ Pembuatan web semantik yang belum umum dan susah, menjadikan web semantik belum populer saat ini. Padahal dengan menggunakan web semantik, informasi-informasi yang ada pada web dapat diolah secara langsung oleh komputer. Penelitian terus dilakukan untuk memudahkan pembuatan web semantik dimana salah satunya adalah Zotonic. Zotonic merupakan salah satu web \textit{framework} yang berbasis semantik. Pada penelitian sebelumnya telah dikembangkan Zotonic yang dapat menerima ontologi sebagai masukan untuk pembentukan struktur webnya. Namun, karena ontologi tidak mengandung \textit{business logic}, maka diperlukan mekanisme untuk menghubungkan ontologi dengan ABS \textit{microservices} agar \textit{business logic} pada Zotonic dapat bersifat dinamis berdasarkan kebutuhkan. Penelitian ini membahas bagaimana ontologi dan \textit{web service} yang dihasilkan oleh ABS \textit{microservices} dapat diintegrasikan. Hal ini berguna agar dapat menciptakan beberapa web pada Zotonic yang memiliki struktur yang sama namun memiliki \textit{business logic} yang berbeda. Tahapan dari penelitian ini adalah melakukan rancangan terhadap integrasi yang akan dilakukan dan implementasi \textit{adaptor} yang digunakan sebagai penghubung antara ontologi dan \textit{web services}. Pada akhir penelitian ini, dilakukan ujicoba untuk melihat hasil dari implementasi yang telah dilakukan.

\vspace*{0.2cm}

\noindent Kata Kunci: \\ 
\noindent ABS, \textit{Adaptor}, Ontologi, SPL, \textit{Web Service}, Zotonic\\ 

\newpage