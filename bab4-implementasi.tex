%-----------------------------------------------------------------------------%
\chapter{\babEmpat}
%-----------------------------------------------------------------------------%

Pada bab ini dijelaskan setiap hal yang dilakukan oleh penulis untuk melakukan implementasi terhadap rancangan yang telah dibuat pada bab sebelumnya. Penulis melakukan implementasi \textit{adaptor} yang akan dipanggil melalui \textit{template engine} dan juga melalui model lainnya. Selain itu, penulis juga memanfaatkan \textit{adaptor} yang sudah diimplementasikan pada \textit{business logic}.

%-----------------------------------------------------------------------------%
\section{\textit{Interface Adaptor}}
%-----------------------------------------------------------------------------%

Untuk dapat menjalankan fungsi \textit{adaptor} yang diinginkan, penulis membuat sebuah model baru pada Zotonic dengan nama m\_abs pada folder \co{root/src/models}. Untuk membuat sebuah \textit{model} pada Zotonic, Zotonic mengharuskan setiap \textit{model} untuk mengekspor beberapa fungsi yang dimiliki oleh Zotonic yaitu m\_find\_value, m\_to\_list, dan m\_value agar fungsi tersebut dapat digunakan melalui \textit{template engine}.

\begin{minipage}{\linewidth}
\begin{lstlisting}[caption={Fungsi yang harus diekspor untuk \textit{model}},label={lst:librarymodel}]
-export([
	m_find_value/3,
	m_to_list/2,
	m_value/2,
]).
\end{lstlisting}
\end{minipage}\\\\

Seperti yang dapat dilihat pada Kode \ref{lst:librarymodel}, agar fungsi tersebut dapat dijalankan pada \textit{template engine}, tentu harus diimplementasi sesuai dengan kebutuhan. Sesuai dengan kebutuhannya agar \textit{adaptor} dapat dipanggil melalui \textit{template engine}, maka perlu didefinisikan terlebih dahulu bagaimana nantinya \textit{adaptor} dipanggil. Pada penelitian ini, penulis mendefinisikan untuk pemanggilan \textit{adaptor} pada \textit{template engine} dilakukan dengan membuat perintah \co{m.abs.namaFungsi[\{query param=value\}]}. Implementasi dari fungsi m\_find\_value yang digunakan untuk melakukan pencocokan pola akan dijelaskan pada bagian berikutnya. Untuk fungsi m\_to\_list karena pada kasus ini tidak digunakan jadi cukup diimplementasikan seperti Kode \ref{lst:mtolist} berikut

\begin{minipage}{\linewidth}
\begin{lstlisting}[caption={Implementasi fungsi m\_to\_list},label={lst:mtolist}]
m_to_list(_, _Context) ->
	[].
\end{lstlisting}
\end{minipage}\\

Sama seperti fungsi m\_to\_list, fungsi m\_value juga tidak dibutuhkan pada implementasi \textit{adaptor} sehingga dapat diimplementasikan seperti Kode \ref{lst:mvalue} berikut

\begin{minipage}{\linewidth}
\begin{lstlisting}[caption={Implementasi fungsi m\_value},label={lst:mvalue}]
m_value(_, _Context) ->
	undefined.
\end{lstlisting}
\end{minipage}

\subsection{Pemanggilan \textit{Adaptor} Melalui \textit{Template Engine}}

Setelah didefinisikan format pemanggilan \textit{adaptor} pada \textit{template engine}, maka akan diimplementasikan pencocokan pola pada proses pemanggilan \textit{adaptor} dengan menggunakan fungsi m\_find\_value. Berikut adalah implementasi dari fungsi m\_find\_value

\begin{minipage}{\linewidth}
\begin{lstlisting}[caption={Implementasi fungsi m\_find\_value},label={lst:mfindvalue}]
% this method to handle call api from template
-spec m_find_value(Key, Source, Context) -> #m{} | undefined | any() when
	Key:: integer() | atom() | string(),
	Source:: #m{},
	Context:: #context{}.

m_find_value(Type, #m{value=undefined} = M, _Context) ->
	M#m{value=[Type]};

m_find_value({query, Query}, #m{value=Q} = _, _Context) when is_list(Q) ->
	[Key] = Q,
	[Url, Param] = lookup_rules(Key),
	case validate_params(Param, Query) of
		false ->
			[{error, "Num of Params not same"}];
		true ->
			{DecodeJson} = fetch_data(binary_to_list(Url), jiffy:encode({Query})),
			lager:info("ABS result : ~p", [DecodeJson]),
			proplists:get_value(<<"data">>, DecodeJson)
	end;

% Other values won't be processed
m_find_value(_, _, _Context) ->
	undefined. 
\end{lstlisting}
\end{minipage}\\

Pada Kode \ref{lst:mfindvalue}, tahap awal untuk mengimplementasikan fungsi m\_find\_value adalah membuat sebuah \textit{specifications} untuk fungsi tersebut. \textit{Specifications} yang merupakan ketentuan yang harus dipenuhi mengenai \textit{input} yang akan diterima oleh fungsi tersebut dan \textit{output} yang akan dihasilkan oleh fungsi tersebut sehingga fungsi akan dijalankan jika dan hanya jika memenuhi dari \textit{specifications} yang telah didefinisikan. Seperti yang sudah didefinisikan bahwa pemanggilan \textit{adaptor} melalui \textit{template engine} dengan cara \co{m.abs.namaFungsi[\{query param=value\}]}, maka ketika dijalankan m\_abs akan menjalankan \co{m\_find\_value(\{query, Query\}, \#m\{value=Q\})} dimana namaFungsi yang didapatkan dari hasil pemanggilan pada \textit{template engine} akan menjalankan fungsi \co{m\_find\_value(Type, \#m\{value=undefined\})} untuk yang akan mengembalikan sebuah \textit{maps} yang akan diterima kembali oleh fungsi \co{m\_find\_value(\{query, Query\}, \#m\{value=Q\})} yang menyimpan \textit{maps} hasil kembalian tersebut ke dalam variabel Q. Lalu parameter yang terdapat pada \textit{template engine} akan disimpan oleh fungsi \co{m\_find\_value(\{query, Query\}, \#m\{value=Q\})} ke dalam variabel Query.

Setelah mendapatkan informasi tentang Query dan Q, fungsi m\_find\_value selanjutnya akan mengambil \textit{key} yang akan dijalankan pada tabel \textit{rules}. Langkah pertama, akan diekstrak \textit{key} yang ingin dijalankan dari Q dengan cara \textit{pattern matching}. Setelah mendapatkan \textit{key} yang diinginkan, \textit{key} tersebut akan dijadikan sebagai input untuk pemanggilan fungsi lookup\_rules yang akan mengembalikan \textit{endpoint} yang akan dipanggil oleh m\_abs serta jumlah parameter dari fungsi yang akan dijalankan. Setelah mendapatkan \textit{endpoint} serta jumlah parameter dari hasil pemanggilan fungsi lookup\_rules yang berbentuk \textit{list}, \textit{list} tersebut akan diekstrak menjadi variabel Url yang menyimpan informasi \textit{endpoint} dan variabel Param yang menyimpan informasi jumlah parameter dari fungsi tersebut. Selanjutnya fungsi m\_find\_value akan memanggil fungsi validate\_params dengan parameter Param yang menyimpan informasi jumlah parameter dan variabel Query yang menyimpan informasi mengenai \textit{list} dari parameter yang diberikan pada \textit{template engine}.

Setelah mendapatkan hasil dari pemanggilan fungsi validate\_params, pada fungsi m\_find\_value akan mengembalikan \textit{error} jika hasil dari pemanggilan fungsi validate\_params bernilai false. Sebaliknya, jika hasil dari pemanggilan fungsi validate\_params bernilai true maka fungsi m\_find\_value akan memanggil fungsi fetch\_data dengan parameter berupa variabel Url dan juga Parameter \textit{Query} yang sudah dijadikan dalam bentuk \textit{json} menggunakan \co{jiffy:encode/1}. Hasil dari pemanggilan fungsi fetch\_data akan mengembalikan sebuah \textit{json} yang berisikan data yang ingin ditampilkan. Sesuai dengan rancangan \textit{web service} yang sudah dijelaskan pada bab sebelumnya, data yang ingin ditampilkan terdapat pada parameter "data" pada \textit{json} sehingga kita perlu mengambil nilai dari parameter "data" tersebut dengan cara \co{proplists:get\_value(<<"data">>, DecodeJson)} dimana \co{DecodeJson} merupakan \textit{json} hasil dari fetch\_data.

\subsection{Pemanggilan \textit{Adaptor} Melalui Model}

Selain dipanggil melalui \textit{template engine}, ada kebutuhan untuk memanggil \textit{adaptor} dari model lain seperti pemanggilannya pada model m\_rsc agar dapat mengirim data dari Zotonic kepada \textit{web service} melalui \textit{adaptor}. Untuk itu, perlu diimplementasi sebuah fungsi baru dengan nama call\_api\_controller pada \textit{adaptor} yang dapat diakses secara langsung oleh model lain seperti berikut

\begin{minipage}{\linewidth}
\begin{lstlisting}[caption={Implementasi fungsi untuk pemanggilan \textit{adaptor} dari model},label={lst:adaptormodel}]
call_api_controller(Key, Data) ->
	[Url, Param] = lookup_rules(Key),
	case validate_params(Param, Data) of
		false ->
			[{error, "Num of Params not same"}];
		true ->
			{DecodeJson} = fetch_data(binary_to_list(Url), jiffy:encode({Data})),
			lager:info("[ABS] result ~p", [DecodeJson]),
			case proplists:get_value(<<"status">>, DecodeJson) of
				200 ->
					proplists:get_value(<<"data">>, DecodeJson),
				201 ->
					Message = proplists:get_value(<<"message">>, DecodeJson),
					lager:info("[ABS] status 201 ~p", [binary_to_list(Message)]);
				400 ->
					Message = proplists:get_value(<<"message">>, DecodeJson),
					lager:error("[ABS] status 400 ~p", [Message]);
				_Other -> 
					lager:error("[ABS] status undefined ~p", [_Other])
			end  
	end.
\end{lstlisting}
\end{minipage}

Pada Kode \ref{lst:adaptormodel}, fungsi call\_api\_controller menerima dua parameter yaitu \textit{Key} dan Data. \textit{Key} merupakan nama fungsi yang ingin dipanggil oleh \textit{web service}, dan Data merupakan data yang akan dikirim ke \textit{web service}. Parameter \textit{Key} yang didapatkan oleh fungsi call\_api\_controller akan digunakan untuk memanggil fungsi lookup\_rules yang akan mengembalikan sebuah \textit{list} yang berisi \textit{endpoint} serta jumlah parameter dari fungsi yang akan dijalankan. \textit{List} tersebut akan diekstrak dimana \textit{endpoint} disimpan ke dalam variabel Url dan jumlah parameter disimpan ke dalam variabel Param. Setelah melakukan ekstraksi, fungsi call\_api\_controller akan memanggil fungsi validate\_params dengan parameter Param yang menyimpan informasi jumlah parameter dan variabel Data yang menyimpan informasi mengenai \textit{list} dari parameter yang ingin dikirim ke \textit{web services}.

Setelah mendapatkan hasil dari pemanggilan fungsi validate\_params, pada fungsi call\_api\_controller akan mengembalikan \textit{error} jika hasil dari pemanggilan fungsi validate\_params bernilai \textit{false}. Sebaliknya, jika hasil dari pemanggilan fungsi validate\_params bernilai \textit{true} maka call\_api\_controller akan memanggil fungsi fetch\_data dengan parameter berupa variabel Url dan juga parameter data yang sudah dijadikan dalam bentuk \textit{json} menggunakan \textit{library} \co{jiffy:encode/1}. Hasil dari pemanggilan fungsi fetch\_data akan mengembalikan sebuah \textit{json} yang berisikan data yang ingin diperoleh. Setelah mendapatkan \textit{json} dari pemanggilan fungsi fetch\_data, dilakukan ekstraksi terhadap parameter "status" dari \textit{json} tersebut. Tujuan dari proses ekstraksi tersebut agar diketahui data yang ingin diambil adalah parameter \textit{"message"} atau parameter "data" dari \textit{json} tersebut. Berikut merupakan tabel \textit{mapping} yang digunakan setelah mendapatkan parameter "status" dari \textit{json}

\begin{table}
	\centering
	\caption{Tabel \textit{Mapping}}
	\label{tab:statuscode}
	\begin{tabular}{| c | l | c |}
		\hline
		Status & Keterangan & Data yang diambil \\ 
		\hline
		200 & Mengambil informasi & Data \\
		201 & Mengeksekusi perintah (\textit{insert, update, delete}) & Message \\
		400 & \textit{Error} dari API & Message \\
		\textit{Other} & \textit{error} & Message \\
		\hline
	\end{tabular}
\end{table}

%-----------------------------------------------------------------------------%
\section{Implementasi \textit{Adaptor}}
%-----------------------------------------------------------------------------%

Sebelum melakukan implementasi \textit{adaptor}, perlu dibuat terlebih dahulu tabel \textit{rules} yang akan dibaca oleh \textit{adaptor}. Namun, karena belum adanya mekanisme pemetaan secara langsung dari \textit{rules} yang dimiliki oleh ontologi ke dalam tabel \textit{rules} maka penulis membuat tabel \textit{rules} secara manual. Tabel \textit{rules} tersebut berbentuk \textit{json} dimana strukturnya sebagai berikut

\begin{minipage}{\linewidth}
\begin{lstlisting}[caption={Struktur tabel \textit{rules}},label={lst:strukturtablerules}]
{
nama fungsi : [endpoint, jumlah parameter]
}
\end{lstlisting}
\end{minipage}

Setelah tabel rules selesai dibuat sesuai dengan struktur pada Kode \ref{lst:strukturtablerules}, tabel rules tersebut disimpan ke dalam sebuah file yang bernama rules.txt dan ditaruh pada \textit{root} dari Zotonic. Selain rules.txt, pada \textit{adaptor} juga perlu diimplementasikan fungsi-fungsi berikut agar \textit{adaptor} dapat bekerja.

\subsection{Implementasi Fungsi \textit{lookup\_rules}}

Fungsi lookup\_rules merupakan fungsi yang membaca berkas rules.txt dan mengembalikan \textit{list} yang berisikan \textit{endpoint} dan jumlah parameter dari \textit{key} yang diberikan sebagai \textit{input}. Adapun implementasi dari fungsi lookup\_rules sebagai berikut

\begin{minipage}{\linewidth}
\begin{lstlisting}[caption={Implementasi fungsi lookup\_rules},label={lst:lookuprules}]
lookup_rules(Key) ->
	File = ?RULES,
	case read_file(File) of
		{error, Error} ->
			[{error, Error}];
		[] ->
			[{error, "File empty"}];
		Json ->
			{DecodeJson} = jiffy:decode(Json),
			proplists:get_value(atom_to_binary(Key, latin1), DecodeJson)
end.

read_file(File) ->
	case file:read_file(File) of
		{ok, Data} ->
			Data;
		eof ->
			[];
		Error ->
			{error, Error}
end.
\end{lstlisting}
\end{minipage}

Pada Kode \ref{lst:lookuprules}, dapat dilihat bahwa fungsi lookup\_rules akan menjalankan fungsi read\_file yang membaca seluruh isi dari file yang telah didefinisikan sebagai tabel \textit{rules}. Perlu didefinisikan juga dimana lokasi dari tabel \textit{rules} yang akan dibaca sebagai variabel final dengan nama variabel \textit{RULES} pada m\_abs seperti Kode \ref{lst:pathrules}
\begin{minipage}{\linewidth}
\begin{lstlisting}[caption={lokasi dari \textit{file} rules.txt},label={lst:pathrules}]
-define(RULES, "./rules.txt").
\end{lstlisting}
\end{minipage}

\subsection{Implementasi Fungsi \textit{validate\_params}}

Implementasi dari fungsi validate\_params dapat dilihat sebagai berikut
\begin{minipage}{\linewidth}
\begin{lstlisting}[caption={Implementasi fungsi validate\_params},label={lst:validateparams}]
validate_params(Param, Query) ->
	case length(Query) == Param of
		false ->
			false;
		true ->
			true
end.
\end{lstlisting}
\end{minipage}

Pada Kode \ref{lst:validateparams}, fungsi validate\_params akan mengembalikan boolean hasil pengecekan jumlah parameter yang didapatkan dari tabel \textit{rules} dan jumlah parameter yang terdapat pada \textit{list} dari \textit{Query}. Kembalikan \textit{true} jika jumlah keduanya sama, dan kembalikan \textit{false} jika jumlahnya tidak sama.

\subsection{Implementasi Fungsi \textit{fetch\_data}}

Adapun implementasi dari fungsi fetch\_data sebagai berikut

\begin{minipage}{\linewidth}
\begin{lstlisting}[caption={Implementasi fungsi fetch\_data},label={lst:fetchdata}]
-spec fetch_data(Url, Query) -> list() when
	Url:: list(),
    Query:: list().
fetch_data(_,[]) ->
    [{error, "Params missing"}];
fetch_data("",_) ->
	[{error, "Url missing"}];
fetch_data(Url, Query) ->
	    case post_page_body(Url, Query) of
	        {error, Error} ->
	            [{error, Error}];
	        Json ->     
	            jiffy:decode(Json)
end.
\end{lstlisting}
\end{minipage}

Pada Kode \ref{lst:fetchdata}, terdapat \textit{specifications} untuk fungsi fetch\_data dimana fungsi ini menerima dua parameter yaitu \textit{url} atau \textit{endpoint web services} yang ingin dituju serta \textit{Query} yang berisikan parameter dalam bentuk \textit{json} yang akan dikirim. Fungsi fetch\_data ini kemudian akan mengembalikan sebuah \textit{list} hasil dari pemanggilan \textit{web services} tersebut. Fungsi fetch\_data akan memanggil fungsi post\_page\_body yang akan menjalankan pemanggilan ke \textit{web service}. Jika fungsi post\_page\_body mengembalikan \textit{Json}, maka \textit{Json} tersebut akan di\textit{decode} menggunakan \co{jiffy:decode/1} sehingga dihasilkan sebuah \textit{list} yang akan dikirim kepada pemanggilan fungsi fetch\_data. Tetapi jika fungsi post\_page\_body mengembalikan \textit{error}, maka fungsi fetch\_data juga akan meneruskan error tersebut kepada fungsi pemanggilnya. Adapun implementasi dari fungsi post\_page\_body sebagai Kode \ref{lst:postpagebody} berikut.

\begin{minipage}{\linewidth}
\begin{lstlisting}[caption={Implementasi fungsi post\_page\_body},label={lst:postpagebody}]
post_page_body(Url, Body) ->
	case httpc:request(post, {Url, [], "application/json", Body}, [], []) of
		{ok,{_, _, Response}} ->
			Response;
		Error ->
			{error, Error}
	end.
\end{lstlisting}
\end{minipage}
%-----------------------------------------------------------------------------%
\section{Penggunaan \textit{Adaptor} pada \textit{Business Logic}}
%-----------------------------------------------------------------------------%

Pada penelitian sebelumnya, proses pembuatan \textit{business logic} dilakukan secara langsung pada \textit{script} datatypePropertyMapper.sh menggunakan \textit{javascript}. Tetapi \textit{business logic} antara satu situs dengan situs lainnya dapat berbeda sehingga dibutuhkan sebuah mekanisme agar \textit{business logic} dapat lebih dinamis berdasarkan ontologi yang digunakan. Untuk mengatasi hal tersebut, maka modul m\_abs yang telah diimplementasi akan digunakan untuk membuat \textit{business logic} agar bersifat dinamis. \textit{Script} datatypePropertyMapper.sh akan menghasilkan \textit{template} untuk admin dan juga \textit{template} untuk halaman pengguna sehingga untuk menambahkan \textit{business logic} yang bersifat dinamis dapat dimanfaatkan implementasi dari pemanggilan \textit{adaptor} dari \textit{template} yang sudah dijelaskan pada bagian sebelumnya. Berikut ini merupakan \textit{business logic} untuk menghitung total sebelum dilakukan \textit{refactoring}.
	
\begin{minipage}{\linewidth}
\begin{lstlisting}[caption={Fungsi total sebelum refactoring},label={lst:totalBefore}]

	var predId = [];
	var total = 0;


	
		
			if (predId.indexOf("{{ pred_id }}") == -1) {
				predId.push("{{ pred_id }}");
		
		
			
				
					total += {{ amountholder.amount }};
				
			
		
		
			}
		
	

\end{lstlisting}
\end{minipage} \\

Pada Kode \ref{lst:totalBefore}, dapat dilihat bagaimana proses penjumlahan untuk fungsi total dilakukan secara \textit{javascript}. Setelah dilakukan \textit{refactoring} dengan cara memanfaatkan implementasi dari pemanggilan \textit{adaptor} dari \textit{template}, maka Kode \ref{lst:totalBefore} akan diganti menjadi Kode \ref{lst:totalAfter}
\begin{minipage}{\linewidth}
\begin{lstlisting}[caption={Fungsi total setelah refactoring}, label={lst:totalAfter}]
{{m.abs.totalDonation[{query id=id}]}}
\end{lstlisting}
\end{minipage} \\

Pada Kode \ref{lst:totalAfter}, akan dipanggil \textit{adaptor} dengan \textit{key totalDonation} yang akan menjalankan fungsi untuk menghitung total donasi dengan id program yang sedang diakses baik pada \textit{template} admin maupun \textit{template} untuk pengguna.
